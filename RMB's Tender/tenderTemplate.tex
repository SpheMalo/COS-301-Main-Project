\documentclass[12pt]{article}

\usepackage{pdflscape}
\usepackage[margin=0.8in]{geometry}
\usepackage{titling}
\usepackage{graphicx} %Required for diagrams
\usepackage[hidelinks]{hyperref}
\usepackage{color}
\usepackage{framed}

\begin{document}

\begin{titlepage}
	\begin{center}
		
		\begin{figure}[t]
			\centering
			\includegraphics[width=350px]{images/UP_Logo.png}
		\end{figure}
		
		% Title
		\textsc{\large Architectural Requirements} \\ 
		\vspace{2cm}
		\textbf{\Huge Project: Figbook} \\ 
		\textsc{\large Client: Figtory Animation} \\ 
		\vspace{2cm}
		\textbf{\Huge Team: Creativate } \\ 
		
		%\begin{minipage}{0.4\textwidth}s
		\begin{flushright} \large
			Armand Pieterse \emph{12167844} \newline
			Kgomotso Sito 		\emph{12243273} \newline
			Jimmy Peleha		\emph{12230830} \newline
			Sphelele Malo 	\emph{12247040} \newline
			Ndivhuwo Nthambeleni 	\emph{10001183} \newline
			\end{flushright}
		%\end{minipage}
		\textsc{\small Department of Computer Science, University of Pretoria}
		
		\vfill
		
	Here's a link to \href{https://github.com/SpheMalo/COS-301-Main-Project.git}{GitHub}.\\
	\url{https://github.com/SpheMalo/COS-301-Main-Project.git}

	\vfill

	{\large \today}	
	
		
		
	\end{center}
\end{titlepage}


\newpage
\tableofcontents

\newpage

\section{The Team}

%Jimmy%
\subsection{full name and surname}

\subsubsection{Interests:}

\subsubsection{Technical Skills:}

\subsubsection{Relevant Past Experience:}

\subsubsection{Non- Technical Strengths:}

\subsubsection{Why I chose this project:}

\newpage
%Sphe%
\subsection{full name and surname}

\subsubsection{Interests:}

\subsubsection{Technical Skills:}

\subsubsection{Relevant Past Experience:}

\subsubsection{Non- Technical Strengths:}

\subsubsection{Why I chose this project:}

\newpage
%Armand%
\subsection{full name and surname}

\subsubsection{Interests:}

\subsubsection{Technical Skills:}

\subsubsection{Relevant Past Experience:}

\subsubsection{Non- Technical Strengths:}

\subsubsection{Why I chose this project:}

\newpage
%Sito%
\subsection{full name and surname}

\subsubsection{Interests:}

\subsubsection{Technical Skills:}

\subsubsection{Relevant Past Experience:}

\subsubsection{Non- Technical Strengths:}

\subsubsection{Why I chose this project:}

\newpage

\newpage

\section{Project Execution}
\subsection{Development Methodology}
	\par{We will be using agile methodology, because it entails the flexibilty needed for such a large-scale data management system. Delivering working aspects of the system will be guaranteed to the client should the implementation be successful.}
\subsection{Client-Team Communication}
	\par{It is absolutely crucial to establish a quality communication line between the team and the client. This is so that there is synchronicity between the team's progress and the client's needs. As specified by the client, meetings will be held on a monthly basis. Any meetings in between will depend on the availability of the cient.}
\subsection{Technical Issues}
	\par{In terms of complexity, implementing a system that is easy to use and provides all the features needed by the client will be challenging. However, we will decouple front-end and back-end to hide away the complexity of the system from the user.}
	\par{The expected influx of data from multiple sources makes it challenging to process the data. It is expected in different formats and a generic accepting feature has to be implemented. Enforcing the ISDA FPML standard will help a great deal in achieving a solution for the aforementioned problem.}
\subsection{Technologies}
	\par{C# will be the language of choice using Visual Studio IDE. As well as all the technologies specified by the client (e.g. Hadoop and mulesoft). As development planning commences, other optimal technologies will be defined to accommodate the needs of the system.}
\subsection{Deliverables}
	\\begin{itemize}
		\item A working system
		\item A working service with each client-team meeting
		\item Source code
		\item Documentation
	\end{itemize}
\end{document}