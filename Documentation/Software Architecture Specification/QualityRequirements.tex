%\documentclass[a4paper,12pt]{report}
%\addtolength{\textwidth}{2cm}
%\addtolength{\topmargin}{-2cm}
%\addtolength{\textheight}{3.5cm}
%\newcommand{\HRule}{\rule{\linewidth}{0.5mm}}
\renewcommand{\labelitemii}{$\star$}
%\begin{document}

\subsection{Quality Requirements}

\begin{itemize}
	\item \textbf{Security}
	\begin{itemize}
		\item The Mediawiki infrastructure provides secure access to the database and the opportunity to modify user interaction with certain pages on the system.
		\item Session management in the underlying has been accounted for. Additional session management is added to restrict unauthorized access to  pages by users.
	\end{itemize}
	\item \textbf{Autidability}
	\begin{itemize}
		\item The Figbook system uses Mediawiki APIs to track changes made my all users on all pages modified. These changes will not be visible to other users except to the administrator and the users working together on manuscripts.
	\end{itemize}
	\item \textbf{Scalability}
	\begin{itemize}
		\item The system uses MySQL in conjuction with MediaWiki APIs, hence the support for both small and large scale use.
		\item The use of objects to represent the database decreases the frequency of expensive calls to the database.
	\end{itemize}
	\item \textbf{Integrability}
	\begin{itemize}
		\item Integrating the system with multiple platforms is not a problem as it is a web-based system.
		\item Figbook has been integrated with Mediawiki and the MySQL DBMS.
	\end{itemize}
	\item \textbf{Usability}
	\begin{itemize}
		\item The Graphical User Interface is quite elegant, simple and user friendly. Underlying architecture and complicated functionality has been abstracted away from the user.
		\item A detailed user manual will be provided.
	\end{itemize}
	\item \textbf{Reliability and Availability}
	\begin{itemize}
		\item The system will provide the user with all the functionality required to carry out the entire lifecycle of a book from the comfort of their own home.
		\item The registration of an ISBN number will be requested from within the system. Failure to obtain an ISBN number is dependent on entities outside the scope of a system.
	\end{itemize}
\end{itemize}


%\end{document}
