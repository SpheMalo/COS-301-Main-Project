

\textbf{Introduction}\paragraph{This section will cover the integration channels of different systems that will eventually make up a working Figbook system.
Access channels will also be discussed. The following are the different channels to access the system together with the APIs used to provide this access. } 

\subsubsection{Integration Channels}
\begin{itemize}
\item Mediawiki - An open source wiki package to be used for Figbook
\begin{itemize}
\item The Mediawiki package will be used as an API within the Figbook project, it will provide collabrative writing previllages this is one the core functionalities of Figbook.
\item The Mediawiki database will be integrated with the Figbook database (both mysql) for uniform persistance.
\item This package will also provide other services like communication, reporting and Authentication.
\item The use of Mediawiki as an API will mean that it will be contained within the system reducing dependencies on external systems.
\end{itemize}
\item SOAP (Simple Access Project Protocol) allows programs that run on different operating system to communicate using HTTP and XML. It is platform and language independent, It is useful in handling asynchronous processing and it supports many protocols and technologies.
\end{itemize}
\subsubsection{Access channels}
\begin{itemize}
\item Figbook will be accessible via a web interface that is bootstraped to cater for all devices with browers.
\item Figbook should be accessible in the future via a Desktop interface as well as a mobile application (not in scope of the project).
\item Access to figbook is provided via the https protocol to ensure security.
\item IPsec(Internet Protocol Security) will allow for a secure IP and to ensure no harmful data is ever transmitted to the servers of figbook.
\end{itemize}

\subsubsection{API Specifications used for integration and access}
\begin{itemize}
\item Web Service Definition Language(WSDL) - will be used to describe the functionality and the operations provided by the web-based service (Figbook system).
\item Interactive Data Language(IDL) - will be used for data handling purposes for the data source integration for Mediawiki and figbook and any other form of data required by the system.
\item Php - Mysql APIs will be used to query the database.
\item Mediawiki uses a RESTful web services API that allows users to access the services without knowledge of the inner system. 
\end{itemize}


