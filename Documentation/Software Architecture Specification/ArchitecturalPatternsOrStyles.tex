%\documentclass[a4paper,12pt]{report}
%\addtolength{\textwidth}{2cm}
%\addtolength{\topmargin}{-2cm}
%\addtolength{\textheight}{3.5cm}
%\newcommand{\HRule}{\rule{\linewidth}{0.5mm}}

%\begin{document}
	\textbf{Introduction}\paragraph{This section will cover the architectural patterns and/or styles we will be using to solve certain challenges we might face, as well as provide us with a way to meet certain quality requirements. } 
	\subsection{Three-tier Architecture}
		\begin{itemize}
			\item Reasons for using:
				\begin{itemize}
					\item Is to allow any of the three tiers to be upgraded or replaced independently, when changes in requirements or technology require such upgrades or replacements. Thus addressing \textbf{maintainability}.
					\item It is an applicable solution for web development. Where the three tiers are divided up into: \begin{itemize}
		\item \textbf{Presentation-tier:} Which will be the front-end of the website.
		\item \textbf{Application-tier:} This is the logical tier which provides the application's functionality.	
		\item \textbf{Data tier:} All the data persistence mechanisms which will be used. 	
	  \end{itemize}
				\end{itemize}
		\end{itemize}
	\subsection{Blackboard Architecture}
\par{The blackboard paradigm defines heterogeneous problem solving representations as independent modules called
knowledge sources(Sections). Knowledge sources can be seen as specialists in sub-felds of the global application and are
only able to solve sub-problems.
Users read and write relevant data in a book(blackboard) which is a structured textual resource existing the the database
\\

Each book or section has a set of triggering conditions that can be satisfied by particular kinds of events, that is
global changes in the book resulting from user inputs. Only when these events are satisfied can you manipulate the given resource(knowlidge source)
to achieve your goal.
}
\begin{itemize}
			\item Reasons for using:
				\begin{itemize}
					\item Reusability. Sections are independent specialists that can be reused in different projects. Reuse
is made easier by the fact that there is no direct communication between knowledge sources. 
\item Changeability and maintainability. High level of modularization and clear separation between control and
domain (i.e. the knowledge sources) makes the maintenance phase easier. 
\item Robustness. The model naturally leads to the definition of alternative knowledge sources to solve(Collaborate) a given sub-problem, in this case sections of a book.
				\end{itemize}
\end{itemize}
	
%\end{document}